%%%%%%%%%%%%%%%%%%%%%%%%%%%%%%%%%%%%%%%%%%%%%%%%%%%%%%%%%%%%%%%%%%%%%%%%%%%%%%%%%%%%%%%%
\chapter{Summary and Outlook}
\label{chapter:Conclusion}
%%%%%%%%%%%%%%%%%%%%%%%%%%%%%%%%%%%%%%%%%%%%%%%%%%%%%%%%%%%%%%%%%%%%%%%%%%%%%%%%%%%%%%%%
%
%
In this thesis a number of three-dimensional tricoordinated lattices were introduced for which the Kitaev honeycomb model could be defined while retaining its exact solvability.
In doing so, a variety of novel gapless \ZZ~spin liquid states were revealed.
The gapless fermionic quasiparticle excitations occurring in these spin liquids vary greatly depending on the underlying lattice, ranging from fully two-dimensional Fermi surfaces and nodal lines protected by winding numbers of different dimension, to point-like topological degeneracies analogous to the Weyl fermion of high-energy theory.

Alongside a detailed analysis of the various three-dimensional gapless spin liquid phases, the concepts of quantum order and the projective symmetry group were utilized to deduce general constraints on the gapless excitations, thereby providing a classification scheme for the gapless Kitaev spin liquids.
A further investigation was undertaken to map the significantly more complicated ground state phase diagram for the non-bipartite lattice (9,3)a which was seen to host a number of both gapped and gapless \textit{chiral} spin liquid phases.
Finally, first results from a study of correlations in Kitaev spin liquids were reported.
Here it was seen how the various Fermi surface topologies of the gapless quasiparticles influence the algebraic correlations of certain four-spin correlation functions.

Beyond the obvious conclusion of the work on algebraic correlations in the Kitaev spin liquids, there are a number of interesting avenues for further study one might consider.
Already, the Kitaev model for several of the lattices introduced in this work have been studied via quantum Monte Carlo simulation~\cite{YasuyukiPRB2017,EschmannPRL2019} as well as in the context of potential experimental signatures via Raman spectroscopy~\cite{PerreaultPRB2016a,PerreaultPRB2016b} and resonant inelastic x-ray scattering~\cite{HalaszPRL2017}.

Whereas the work presented in this thesis focuses heavily on the gapless Kitaev spin liquids, a thorough investigation of the \textit{gapped} spin liquids might also be promising.
It has long been known that the gapped spin liquid phase of Kitaev's original model on the honeycomb lattice is equivalent to the toric code model and hosts exotic Abelian anyon quasiparticles~\cite{KitaevAoP2006}.
A study of the gapped spin liquid phase for the hyperhoneycomb lattice~\cite{MandalPRB2014} uncovered loop-like excitations obeying fermionic statistics.
Furthermore, it was shown that a fermionic loop braided through a larger loop excitation acquires a non-trivial phase.
In Appendix~\ref{appendix:LoopModels} a number of effective loop models were derived for the gapped three-dimensional Kitaev spin liquids, however, the nature of their excitations remains an open question.

Finally, one might investigate the role of disorder in three-dimensional Kitaev spin liquids.
One potential source of disorder is the inclusion of vacancies.
In fact, isolated (and pairs of) vacancies have been studied before for Kitaev spin liquids on the honeycomb~\cite{WillansPRL2010,WillansPRB2011,SanthoshPRB2012,HalaszPRB2014} and hyperhoneycomb~\cite{SreejithPRB2016} lattices.
On the two-dimensional honeycomb lattice it was shown that a single vacancy binds a flux and induces a local moment.
In the gapped phase this moment is free, whereas in the gapless phase the low-field vacancy magnetization is given by $m(h) \sim h \ln{(1/h)}$, where $h$ is a weak, applied magnetic field.
For the three-dimensional hyperhoneycomb lattice it was found that, while a vacancy does not bind a \ZZ~flux, it similarly induces a local moment.
The local moment is again free in the gapped phase, however, in the gapless phase the low-field vacancy magnetization is now given by $m(h) \sim 1/\sqrt{\ln{(1/h)}}$.
As it is the interaction of the vacancy moment with the surrounding gapless spin liquid which suppresses the magnetization, it would be interesting to explore this problem on other three-dimensional lattices which host qualitatively different nodal manifolds.

Another particularly interesting line of investigation is on the effects of Gaussian spatially-correlated disorder of the exchange couplings in the Weyl spin liquids.
A similar situation has been explored in the Weyl \textit{semi-metals} with a correlated on-site disorder potential.
The resulting physics is seen to be a function of both disorder strength and the correlation length $\xi$ of the disorder potential.
The natural length scale $\lambda$ determined by the momentum-space distance between Weyl nodes allows one to distinguish between disorder with and without substantial backscattering, corresponding to $\xi \ll \lambda$ and $\xi \gtrsim \lambda$, respectively.

For disordered Weyl semi-metals without backscattering and a sufficiently short-ranged random potential, renormalization group calculations~\cite{SyzranovPRB2015} indicate a phase transition from the Weyl semi-metal to a diffusive metal occurs at a finite critical disorder strength.
In this case, the disorder-averaged density of states may be used as an order-parameter for the transition.
Weak disorder is seen to be irrelevant and only renormalizes the Fermi velocity, leaving the quadratic scaling of the density of states intact.
In the vicinity of the transition, the density of states scales linearly with the absolute value of the energy before the system is driven into the diffusive metal phase with a finite density of states.
Missing from the above analysis, however, are the effects of rare-regions in the disorder potential.
Quasilocalized states due to rare, but strong fluctuations of the disorder potential lead to an exponentially small density of states resulting in an avoided quantum critical point~\cite{NandkishorePRB2014,PixleyPRX2016}.
Although these rare-region effects round out the non-analyticity for the smallest energies, the critical scaling can still be observed for sufficiently large energies.

For systems without backscattering and an extremely slowly-varying Gaussian random potential, the system undergoes a transition from a Weyl semi-metal to a diffusive metal at a finite disorder strength and the criticality is controlled by a long-ranged fixed point~\cite{LouvetPRB2017}.
The physical consequences of this fixed point have yet to be explored, however, it is likely that the critical behavior of the density of states is rounded out by rare-region effects as in the case of short-ranged disorder.

For the case of ultra-short-ranged disorder where significant backscattering couples different Weyl nodes, it has been argued that the same results from the analysis of \textit{decoupled} Weyl points with a short-ranged potential still apply for sufficiently weak disorder strengths, \ie, the system undergoes a transition from a Weyl semi-metal to a diffusive metal at some finite critical disorder strength~\cite{PixleyPRB2017}.
However, for even stronger disorder, strong backscattering between Weyl points drives the system through an Anderson localization transition~\cite{PixleyPRL2015}.

An interesting question is how the physics of disordered Weyl \textit{spin liquids} might differ.
As the Weyl spin liquids fall into different Altland-Zirnbauer symmetry classes than the Weyl semi-metals, it is not unreasonable to expect the spin liquids to be affected differently by disorder.
There are also the effects of the \ZZ~gauge field to consider.
On the one hand, sign changes in hopping strengths of the Majoranas may be gauged away for sufficiently weak disorder.
On the other hand, strong disorder may drive the ground state into more complex flux sectors, significantly complicating such a study.