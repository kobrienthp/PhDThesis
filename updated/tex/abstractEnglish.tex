%%%%%%%%%%%%%%%%%%%%%%%%%%%%%%%%%%%%%%%%%%%%%%%%%%%%%%%%%%%%%%%%%%%%%%%%%%%%%%%%%%%%%%%%
\chapter*{Abstract}
%%%%%%%%%%%%%%%%%%%%%%%%%%%%%%%%%%%%%%%%%%%%%%%%%%%%%%%%%%%%%%%%%%%%%%%%%%%%%%%%%%%%%%%%
%
%
The formation of quantum spin liquids in frustrated magnets represents an exciting possibility due to the rather exotic features they harbor, including fractionalized excitations and emergent gauge fields.
Unfortunately, they are notoriously difficult to study as there are often no good analytical methods available and quantum Monte Carlo simulations are hindered by the negative sign problem.
The Kitaev honeycomb model is a notable exception of a frustrated quantum model which is exactly solvable and which hosts a number of distinct quantum spin liquid ground states.
As such, it allows for a rare opportunity to study the physics of spin liquids with full analytical control.

In this thesis, we study the fractionalization of spin-1/2 moments into Majorana fermions and an emergent \ZZ~gauge field in a generalization of the Kitaev honeycomb model to a number of three-dimensional lattices.
While the excitations of the gauge field are always gapped, the fermionic quasiparticles may exhibit a gapless dispersion, forming fully two-dimensional Fermi surfaces, symmetry protected nodal lines, or topological Weyl nodes.
We show that one can deduce rather general constraints on the possible gapless excitations by making use of an object called the \textit{projective symmetry group}.
In doing so we provide a scheme for classifying the various gapless Kitaev spin liquids.
A thorough analysis is carried out for a number of these spin liquids, primarily investigating the stability of the gapless modes and the novel features resulting from their sometimes non-trivial topology, as well as their effects on certain equal-time correlation functions.