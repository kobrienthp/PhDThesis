%%%%%%%%%%%%%%%%%%%%%%%%%%%%%%%%%%%%%%%%%%%%%%%%%%%%%%%%%%%%%%%%%%%%%%%%%%%%%%%%%%%%%%%%
\chapter*{Kurzzusammenfassung}
%%%%%%%%%%%%%%%%%%%%%%%%%%%%%%%%%%%%%%%%%%%%%%%%%%%%%%%%%%%%%%%%%%%%%%%%%%%%%%%%%%%%%%%%
%
%
Die Bildung von Quanten-Spinfl\"ussigkeiten in frustrierten Magneten stellt eine aufregende M\"oglichkeit dar, da sie ziemlich exotische Eigenschaften aufweisen, wie z. B. fraktionalisierte Anregungen und auftauchende Eichfelder.
Leider sind sie bekannterma\ss en schwer zu untersuchen, da oft keine guten Analysemethoden verf\"ugbar sind und Quanten-Monte-Carlo-Simulationen durch das Problem der negativen Vorzeichen nicht verwendbar werden.
Das Kitaev-Modell ist eine bemerkenswerte Ausnahme eines frustrierten Quantenmodells, das genau l\"osbar ist und eine Reihe unterschiedlicher Quanten-Spinfl\"ussigkeit-Grundzust\"ande enth\"alt.
Daher bietet es die seltene Gelegenheit, die Physik von Spinfl\"ussigkeiten mit voller analytischer Kontrolle zu untersuchen.

In dieser Arbeit untersuchen wir die Fraktionalisierung von Spin-1/2-Momenten in Majorana-Fermionen und ein aufkommendes \ZZ-Eichfeld in einer Ver\-all\-ge\-mei\-ne\-rung des Kitaev-Modells auf eine Anzahl dreidimensionaler Gitter.
W\"ahrend die Anregungen des Eichfeldes immer gegapped sind, k\"onnen die fermionischen Quasiteilchen eine gapless Dispersion aufweisen, die vollst\"andig zweidimensionale Fermi-Fl\"achen, wegen Symmetrie gesch\"utzte Knotenlinien oder topologische Weyl-Knoten bildet.
Wir zeigen, dass man eher allgemeine Einschr\"ankungen f\"ur die m\"oglichen gapless Anregungen ableiten kann, indem man ein Objekt verwendet, das als \textit{Projective Symmetry Group} bezeichnet wird.
Damit stellen wir ein Schema zur Klassifizierung der verschiedenen gapless Kitaev-Spin-Fl\"ussigkeiten bereit.
F\"ur eine Anzahl dieser Spin-Fl\"ussigkeiten wird eine gr\"undliche Analyse durchgef\"uhrt, wobei vor allem die Stabilit\"at der gapless Moden und die neuen Merkmale untersucht werden, die sich aus ihrer manchmal nicht-trivialen Topologie ergeben, sowie deren Auswirkungen auf bestimmte Korrelationsfunktionen.